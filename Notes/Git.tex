\documentclass{book}
\title{\Huge Git}
\author{Giordano AF} 
\date{}
\begin{document}
	
	\maketitle

	Git instala un bash predefinido para usar los comandos de git. \\
	Estados dentro de git:
\begin{enumerate}
	\item Working Directory $\rightarrow$ los archivos que actualmente trabajo. Para pasar al siguiente estado se ingresa el comando git add ...
	\item Staging Area $\rightarrow$  los archivos que serviran para una primera version. Para pasar al siguiente estado de ingresa el comando git commit ...
	\item Repository $\rightarrow$ los archivos que ya estan finalizados para la version final
\end{enumerate}
	Comandos m\'as usados
\begin{enumerate}
	\item git init \\
	Es una manera de decir que se va a inicar un proyecto nuevo en git o vas a empezar a trabajar un proyecto con git.
	\item git add $<$file$>$ \\
	Para pasar un file de working directory a staging area
	\item git status \\
	Para visualizar los estados de los archivos
	\item git commit $<$file$>$ git commit -m "message" $<$file$>$\\
	Para pasar a un files a repository para un primer snapshot
	\item git push \\
	Subir los files a servidor para que los demas desarrolladores con permiso puedan verlos.
	\item git pull \\
	Traer los cambios de los otros desarrolladores 
	\item git clone \\
	Clonar o copiar	el proyecto desde el servidor hasta tu equipo
	\item git log\\
	Primer ver los cambios historicos que se dieron al proyecto, se puede ver los snapshots.
	\item git checkout -- $<$file$>$\\
	revierte los cambios en el  $<$file$>$
	\item git diff $<$file$>$\\
	Permite ver las diferencias en $<$file$>$
	\item git branch ``branchname''\\
	Permite crear una rama
	\item git branch \\
	Permite visualizar las ramas existentes.
	\item git checkout ``branchname''login \\
	Permite cambiar de rama
	
\end{enumerate}
Observaciones::\\
Antes de hacer un commit es necesario configurar los registros de autenticacion para ello seguimos los siguientes comandos:\\
\begin{itemize}
	\item git config --global user.email ...
	\item git config --global user.name ...
\end{itemize}
Si hay algun problema al tipear mas los datos, simplemente vuelve a ejecutar el comando con la configuracion correcta.\\

Cuando queremos que ciertas carpetas y/o archivos sea ignorados por git, entonces creamos la carpeta .gitignore y colocamos todo lo que queremos que sea ignorado por git dentro de esta carpeta.\\
No olvidar agregar(git add) y commmit .gitignore, ya que esta carpeta si debe ser agregado y comitiado a git.\\

Una vez tengamos nuestro proyecto avanzado podemos subirlo a un repo de github. Seguimos los siguientes pasos:\\
\begin{itemize}
	\item git init en la carpeta que esta mos trabajando
	\item git add README.md
	\item git commit -m ``message''
	\item git remote add origin $<$el enlace de nuestro repo en github$>$
	\item git push -u origin master,aqui agregamos lo que tenemos en la rama master de github
\end{itemize}
\end{document}
